\documentclass[12pt, a4paper, oneside]{article}
\usepackage[english]{babel}
\usepackage{xcolor,listings}
\usepackage{hyperref}
\usepackage{mathtools}
\usepackage{amsmath}
\usepackage{datetime}
\usepackage{graphicx}
% For fitting table to page
\usepackage{adjustbox}
% For pseudocode
\usepackage{algorithm}
\usepackage[noend]{algpseudocode}
\makeatletter
\def\BState{\State\hskip-\ALG@thistlm}
\makeatother
%

\setlength{\parindent}{0pt}

\begin{document}
\title{ECE 495/595: Speech Analysis Project Management Plan}
\author{
  Grace, Jayson \\
  \texttt{jaysong@unm.edu}
  \and
  Rutherford, Amber \\
  \texttt{anhruthe@gmail.com}  
  \and
  Sampelli, Meghana \\
  \texttt{meghana.sampelli2811@gmail.com}
  \and
  Kim, Nicole \\
  \texttt{nkim0912@salud.unm.edu}
  \and
  Gordon, Jeffrey \\
  \texttt{jeffreyrgordon@gmail.com}
  \and
  Dara, Harish \\
  \texttt{harish225@unm.edu} 
}
\date{\today}%
\maketitle

\pagenumbering{gobble} 
\pagebreak
\pagenumbering{arabic}

\section*{Initial Project Plan}
To ensure our success on the project, we have laid out a series of milestones and meetings to discuss progress, road blocks and successes.

\subsection*{Milestones}

We will ensure that we are able to meet milestones laid out on the project specification provided by the professor.

\subsection*{Anticipated Meetings}

We will meet twice a week, once on Wednesday and once on Saturday. This will give us the ability to stay up-to-date on the progress of each member in the group.

\section*{Anticipated Team Member Roles}
There are a total of six people on our team. The responsibility breakdown of every member of the team is listed as a contract for each individual members participation within the scope of the project.

\subsection*{Amber Rutherford}
Amber has background knowledge in backend web development. In order to leverage her skillset properly, we will use her expertise to contribute to both the controller and model components of the project.  She will also be contributing javascript work to the frontend.

\subsection*{Meghana Sampelli}
Meghana has worked extensively with CSS and HTML throughout her technical career. Subsequently she will be focused primarily on the frontend component of the web application.

\subsection*{Nicole Kim}
Nicole has experience with backend database work concerning web development. As a result, she will be contributing to the Model and Controller components of the project. She will also lead efforts on client feedback and interaction initiatives to ensure that we are able to meet the client's needs.

\subsection*{Jeffrey Gordon}
Jeff has substantial experience in working with Javascript and CSS in a frontend capacity. He will be the team lead for the frontend component of the project. He will also ensure that proper spec tests are created for the frontend work to ensure consistency of the front end components.

\subsection*{Jayson Grace}
Jayson has extensive experience in ruby and will be contributing to various tasks involved with the project. He will also be working as the project manager. As a result, he will ensure deadlines are met and team members are on task. Help team to delegate tasks among individuals involved with project. 

\subsection*{Harish Dara}
Harish works as a DBA with both Oracle and Microsoft technologies. He will be contributing to the backend architecture as well as the implementation of proper validations. He will be sure that requirements are met. He will also be in charge of ensuring that meeting times are kept and everyone is aware of changes in schedule.   

\section*{Technology Stack}
We will leverage multiple libraries (gems) to accomplish this project, as well as several javascript libraries for frontend work. The definitive list of technologies is as follows: \\

\textbf{Ruby:}
\begin{itemize}
\item ERB
\item haml
\item Devise
\item SCSS
\item simpleform
\item bootstrap
\item PostgreSQL
\item SQLite \\
\end{itemize}
\textbf{JavaScript:}
\begin{itemize}
\item jQuery
\item CoffeeScript
\item ReactJS
\end{itemize}

\section*{Project Repositories}
The big components of the project repositories will be hosted on github either in the README.MD or in the docs folder of the project there.  \\

We will be utilizing pull requests to do individual work and manage access to the master repository.

\subsection*{Requirements}
Our requirements for this project are as follows:

\begin{itemize}
\item Input a speaker, and have information associated with that speaker.
\item Associate a phoneme with a letter or cluster of letters.
\item Query various speaker characteristics
\item Associate speakers characteristics with phonemes
\end{itemize}

\subsection*{Design Documents}

This will encompass the following items: 

\begin{itemize}
\item API Docs - Outlines the various components within the code.
\item User Docs - Outlines how the program works from the perspective of a user.
\item UML Diagram - Outlines how the database schema comes together and the various relationships that exist between the models.
\end{itemize}

\subsection*{Code repository URL}
The code for this project will be hosted at this github repository: \url{https://github.com/l50/speech_analysis} and at various forks of this repository for each team member. There will be a strict repository accounting system that utilizes a develop branch for development cycles of the software and a master branch for the production ready version. \\

Each member of the team will have their own version of the repository and will issue pull request whenever they have code that is ready to be checked into the primary repository.

\subsection*{Project Management Tools}
In order to aggregate all requirements for the work to be done, we will be using Doodle for scheduling and Skype for remote meetings.
\end{document}